\section{선행 연구}
\ganadareset

%%%%%%%%%%%%%%%%%%%%%%%%%%%%%%%%%%%%%%%%%%%%%%%%%%%%%%%%%%%%%%%%%%%%%%%%%%%%%%%%%%%
\ganada 전영석(2024)의 연구
%%%%%%%%%%%%%%%%%%%%%%%%%%%%%%%%%%%%%%%%%%%%%%%%%%%%%%%%%%%%%%%%%%%%%%%%%%%%%%%%%%%

전영석(2024)은 과학 우수 학생을 대상으로 온도 측정 과정에서의 오차 인식과 불확실성 이해를 분석하였다.
%
연구자는 과학 실험에서 측정과 오차가 탐구의 신뢰성을 결정짓는 핵심 요소임에도,
교육 현장에서 이를 단순 계산 절차로만 다루는 한계를 지적하였다.
%
이를 위해 중등과학올림피아드 예비대표단과 대표단 총 20명을 대상으로 물의 온도 측정 실험을 수행하게 하고,
측정 태도와 오차 인식 과정을 관찰하였다.
%
학생들은 온도계 판독이나 측정 절차를 알고 있었으나 실제 상황에서는 이를 일관되게 적용하지 못했으며,
측정값의 변동을 오차로 해석하지 못했다.
여러 측정값이 주어질 때 대부분 평균만 계산하여 대표값으로 제시하였고,
불확실성이나 유효숫자 적용은 거의 이루어지지 않았다.
오차의 원인을 도구 결함이나 실험자의 실수 등 외적 요인으로만 인식하는 경향도 나타났다.

연구자는 학생들의 인식 수준을 절차적(procedural), 개념적(conceptual), 주체적(agentic)
단계로 구분한 결과, 모든 학생이 절차적 수준에 머물러 있음을 보고하였다.
%
즉, 주어진 지시에 따라 측정과 계산을 수행하지만 오차의 원인과
결과의 신뢰도를 스스로 해석하지 못하는 수준이었다.
%
이러한 결과를 바탕으로, 연구자는 불확실성 중심의 실험 교육과 반복 측정 및 자료 해석
경험의 강화가 필요하다고 제언하였다.
%
본 연구는 과학 우수 학생들조차 오차를 절차적으로만 인식한다는 점을 실증적으로 보여줌으로써,
측정 개념 교육의 개선 방향을 제시한 선행연구로 의의가 있다. \\



%%%%%%%%%%%%%%%%%%%%%%%%%%%%%%%%%%%%%%%%%%%%%%%%%%%%%%%%%%%%%%%%%%%%%%%%%%%%%%%%%%%
\ganada 이재봉(2006)의 연구
%%%%%%%%%%%%%%%%%%%%%%%%%%%%%%%%%%%%%%%%%%%%%%%%%%%%%%%%%%%%%%%%%%%%%%%%%%%%%%%%%%%

이재봉(2006)은 학생들의 측정과 오차 개념을 탐색하며,
과학 탐구 과정에서 불확실성 개념이 어떻게 작용하는지를 연속적으로 분석하였다.
%
첫 번째 연구인 「측정 자료의 오차와 불확실도에 대한 학생들의 이해」에서는
고등학생과 대학생을 대상으로 측정 자료의 표현, 정확도와 정밀도의 구별, 오차의 원인, 불확실성의 전파 등
네 영역의 문항을 통해 학생들의 개념 수준을 조사하였다.
%
학생들은 측정값을 평균으로 단순화하면서 자료의 분포나 변동성을 고려하지 않았고,
불확실성을 단순 계산 절차로 인식하였다.
%
또한 정밀도와 정확도를 혼동하거나 계통 오차와 우연 오차의 차이를 명확히 구분하지 못하였으며,
오차의 원인을 실험 도구나 개인의 실수로 한정하는 경향을 보였다.
%
이재봉은 학생들이 측정값을 과학적 증거로 해석하는 능력이 부족하다고 지적하며,
불확실성과 오차의 구분, 측정의 신뢰도 해석을 중심으로 한 교육이 필요함을 제안하였다.

또 다른 연구인 「학생들의 측정불확실도 개념의 결핍으로 인한 물리탐구과정에서의 어려움 분석」에서는
예비 중등교사 27명을 대상으로 진자 운동 실험을 수행하게 하고,
불확실도 개념의 부족이 탐구 수행 전반에 미치는 영향을 분석하였다.
%
학생들은 실험 설계 단계에서 변인 통제보다 반복 측정에 초점을 두거나,
불확실도의 크기를 고려하지 않은 채 결과를 도출하는 등 과학적 타당성을 확보하지 못하였다.
자료 해석 과정에서도 오차 막대나 추세선을 활용하지 못하고,
그래프를 단순히 시각적으로 해석하는 수준에 머물렀다.
%
이로 인해 자료의 변동을 오차와 구별하지 못하고, 불확실도를 반영한 결론 도출에도 실패하였다.
%
연구자는 이러한 문제의 원인을 결과 중심의 학습 관행에서 찾으며,
측정의 분포와 변동을 탐구의 일부로 인식할 수 있도록 지도해야 한다고 강조하였다. \\



%%%%%%%%%%%%%%%%%%%%%%%%%%%%%%%%%%%%%%%%%%%%%%%%%%%%%%%%%%%%%%%%%%%%%%%%%%%%%%%%%%%
\ganada 지영래(2019)의 연구
%%%%%%%%%%%%%%%%%%%%%%%%%%%%%%%%%%%%%%%%%%%%%%%%%%%%%%%%%%%%%%%%%%%%%%%%%%%%%%%%%%%

지영래(2019)는 사범대학 물리교육과 학부생들이 수행한 작은 질량 측정 실험 보고서를 분석하여,
대학생의 측정과 오차 이해 수준을 구체적으로 탐색하였다.
연구 대상은 물리교육 전공 2학년생 25명으로,
이들은 일상적 재료를 활용해 작은 질량 측정 도구를 제작하고,
측정 과정과 오차 요인, 유효숫자 처리, 측정값의 신뢰도 판단 등을 보고서 형태로 기술하였다.
%
연구의 목적은 학생들이 실험을 통해 측정 원리와 오차 개념을 어떻게 이해하고 적용하는지를 분석하는 데 있었다.
%
학생들은 측정 도구의 개발 원리를 정성적으로는 설명했으나, 이를 수치적, 분석적 모델로 표현하지 못하였다.
측정 과정에서는 반복 측정의 필요성을 인식하지 못하고 대부분 5회 미만의 시도에 그쳤으며,
모눈종이의 면적을 한 번만 측정해 `정확한 값'으로 간주하는 등 참값 중심(point paradigm)의
사고 경향을 보였다.
%
유효숫자 처리 또한 미숙하여, 동일한 도구로 측정했음에도 자리수를 다르게 기록하거나
평균 계산 시 규칙을 지키지 못하는 오류가 다수 확인되었다.
%
계통오차와 우연오차의 구분에서도 혼동이 나타나, 측정 과정의 불안정성을 우연오차로,
도구의 부정확성을 계통오차로 구분하지 못하는 사례가 많았다.
%
학생들이 제시한 계통오차 요인은 각도 측정의 어려움, 측정 도구 눈금의 부정확함,
핀과 지지대 사이의 마찰 등이었으며, 우연오차 요인은 외력, 관찰자의 움직임,
측정자의 숙련도 부족 등이었다.
%
그러나 두 오차의 개념을 혼동하는 응답이 절반 이상을 차지해,
오차 분류의 기준과 과학적 근거가 명확히 내면화되지 않았음을 보여주었다.
%
측정값의 신뢰 여부를 판단할 때는 `참값과의 오차가 작다'거나 `실험을 열심히 수행했다'는
비과학적 근거를 제시하는 경우가 많았다.
%
연구자는 이러한 결과를 바탕으로, 대학 수준에서도 학생들이 측정과 오차 개념을
형식적 규칙으로만 이해하고 탐구 맥락 속에서 기능적으로 활용하지 못하고 있음을 지적하였다.
또한 측정 탐구 요소를 중심으로 한 정교한 실험 설계와 보고서 환류가 필요하다고 제언하였다.