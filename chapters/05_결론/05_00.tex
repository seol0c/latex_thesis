
본 연구는 과학영재 학생들이 측정과 자료 해석 과정에서 유효숫자 규칙을
얼마나 일관되게 적용하는지를 분석하였다.
%
그 결과, 도구나 눈금 간격이 명시된 문항에서는 60~70\%의 학생이 유효숫자 규칙을 바르게 답했으나,
실제 측정과 계산 상황에서는 8~30\% 수준만이 이를 적용하여,
유효숫자 지식과 실제 적용 사이의 뚜렷한 격차가 나타났다.
%
즉, 학생들은 이론적으로는 유효숫자 규칙을 이해하고 있으나,
실제 탐구 상황에서는 그 의미를 반영하지 못한 채 절차적으로 처리하는 경향을 보였다. \\

또한 응답 유형 분석 결과, 유효숫자 규칙을 알고도 생략하거나
단순화하는 학생이 전체의 약 80\%를 차지하였으며,
자릿수를 일관성 있게 사용하는 `일관형'보다 규칙을 알고도 무시하거나 간과하는 `간과형'이 우세하였다.
%
이는 학생들이 측정 도구의 정밀도와 불확실도를 고려하지 않은 채,
`눈금 단위까지만 기록한다'는 단순 규칙에 의존하고 있음을 보여준다.

%% 표. 측정과 자료 해석에서 정답률 차이
\MeasurementVsInterpretationGap

유효숫자의 필요성 인식에서도 정밀한 도구인 버니어캘리퍼스 상황에서는 81\%의 학생이 필요성을 인식했으나,
자의 경우 64\%만이 필요하다고 응답하였다.
%
즉, 도구의 정밀도가 높을수록 유효숫자를 중요하게 여겼지만,
일상적이고 정밀하지 않은 상황에서는 그 필요성을 낮게 평가하는 인식의 불안정성이 존재하였다.
일부 학생은 `물리 상수는 이미 정확하다'거나
`일상생활에서는 굳이 지킬 필요가 없다'는 이유를 제시하여,
유효숫자의 기능을 불확실성과 신뢰도의 표현으로 이해하지 못하는 모습을 보였다. \\

이러한 결과는 학생들이 유효숫자 개념을 규칙 차원에서는 알고 있으나,
과학적 맥락 속에서 그 필요성과 기능을 충분히 내면화하지 못하고 있음을 시사한다.
%
다시 말해, 지식 수준에서는 이해하지만 실제 탐구 과정에서는 일관된 판단으로 연결되지 못하고 있으며,
이는 탐구적 사고의 단절로 이어지고 있다. \\

따라서 유효숫자 교육은 단순한 규칙 전달을 넘어,
측정 도구의 정밀도와 실험 상황을 연계하여 유효숫자의 필요성과 의미를
체감적으로 이해하도록 하는 방향으로 보완될 필요가 있다.
%
학생들이 다양한 측정 사례를 통해 불확실성과 정밀도의 관계를 탐색하고,
측정값의 기록, 계산, 해석 전 과정에서 자릿수의 일관성을 스스로 점검하도록 하는
교수학습 설계가 요구된다.
%
이러한 접근을 통해 유효숫자는 더 이상 `표기 규칙'이 아닌
과학 탐구의 신뢰성을 유지하는 사고의 틀로 정착될 수 있을 것이다.