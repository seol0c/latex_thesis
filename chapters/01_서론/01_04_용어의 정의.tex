\section{용어의 정의}
\ganadareset
\ganada 유효숫자

유효숫자는 측정값을 수치로 표현할 때 그 수치가 지닌 신뢰도를 나타내는 자리수로,
측정 도구의 정밀도와 측정자의 판독 습관 등을 반영한다.
본 연구에서는 유효숫자를 `측정 도구의 최소 눈금 1/10까지 판독하여 기록한 수치가 지닌 신뢰 가능한 자리수'로
정의하며, 학생들이 이러한 기준을 실제 측정과 계산 과정에서 얼마나 일관되게 적용하는지를 분석하였다. \\

\ganada 불확실성

불확실성은 측정 결과가 참값으로부터 얼마나 벗어나 있을 수 있는지를 나타내는 양으로,
측정 도구의 한계와 측정자의 판단 오차를 포함한다.
본 연구에서는 불확실성을 유효숫자의 근거가 되는 개념으로 간주하며,
학생들이 측정값의 자릿수를 결정하거나 평균을 계산할 때 불확실성을 얼마나 인식하고 반영하는지를 평가하였다. \\

\ganada 측정

측정은 물리량을 표준 단위와 비교하여 수치로 표현하는 행위로,
도구의 최소 눈금 단위와 판독 방법에 따라 결과의 신뢰도가 달라진다.
본 연구에서는 자 또는 버니어캘리퍼스 등 측정 도구를 이용하여 길이를 측정하는 과정을
`측정 상황'으로 설정하고, 학생들이 도구의 눈금 간격과 판독 원리를 근거로 유효숫자를
올바르게 결정하는지를 분석하였다. \\


편집중




\ganada 자료 해석

자료 해석은 측정된 여러 수치를 비교, 종합하여 의미 있는 결론을 도출하는 과정이다.
본 연구에서는 반복 측정값의 평균 계산을 중심으로 학생들이 유효숫자 규칙을 적용하고
계산된 결과의 자릿수를 적절히 표현하는지를 확인하였다. \\

\ganada 유효숫자 필요성 인식

유효숫자 필요성 인식이란 측정 과정뿐 아니라 자료 처리,
상수 적용 등 다양한 탐구 맥락에서 유효숫자가 왜 필요한지를 자각하는 정도를 의미한다.
본 연구에서는 자, 버니어캘리퍼스, 물리 상수 등 서로 다른 정밀도의 도구나 상황에 따라
학생들의 인식이 어떻게 달라지는지를 분석하였다.







