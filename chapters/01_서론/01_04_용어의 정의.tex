\section{용어의 정의}
\ganadareset
\ganada \hly{측정과 불확도

`고등학교 물리학 실험 교과서(박종찬 외. 2024)'에서 측정은
`적절한 도구나 기계를 사용하여 길이, 넓이 등 단위로 나타낼 수 있는
물리량을 재어서 기록하는 활동'으로 설명하고,
측정의 목적은 수학적으로 표현할 수 있는 과학적 법칙이나 원리의 기초 자료를 얻는 데 있음을 강조하였다.
또한 측정에는 항상 오차(error)가 포함되고, 실제 실험 상황에서 참값을 정확히 알 수 없으므로
`불확도(uncertainty)'를 구하여 측정값을 표현한다고 설명한다.}
%
측정은 도구의 최소 눈금 단위와 판독 방법에 따라 결과의 신뢰도가 달라진다.
본 연구에서는 자 또는 버니어캘리퍼스 등 측정 도구를 이용하여 길이를 측정하는 과정을
`측정 상황'으로 설정하고, 학생들이 도구의 눈금 간격과 판독 원리를 근거로 유효숫자를
올바르게 결정하는지를 분석하였다.
%
불확도는 측정 결과가 참값으로부터 얼마나 벗어나 있을 수 있는지를 나타내는 양으로,
측정 도구의 한계와 측정자의 판단 오차를 포함한다.
본 연구에서는 불확도를 유효숫자의 근거가 되는 개념으로 간주하며,
학생들이 측정값의 자릿수를 결정하거나 평균을 계산할 때 불확도를
얼마나 인식하고 반영하는지를 확인하고자 하였다. \\

\ganada 유효숫자\hly{와 불확도의 결정 방법}

유효숫자는 측정값을 수치로 표현할 때 그 수치가 지닌 신뢰도를 나타내는 자리수로,
측정 도구의 정밀도와 측정자의 판독 습관 등을 반영한다.
본 연구에서는 유효숫자를 `측정 도구의 최소 눈금 1/10까지 판독하여 기록한 수치가 지닌 신뢰 가능한 자리수'로
정의하며, 학생들이 이러한 기준을 실제 측정과 계산 과정에서 얼마나 일관되게 적용하는지를 분석하였다.

\hly{
`고등학교 물리학 실험 교과서(박종찬 외. 2024)'에서
대푯값을 표현하는 규칙으로 유효숫자는 불확도와 같은 소수점 자리가 되도록 반올림하여 표현해야함을 언급하였고
불확도의 결정 방법에 대하여 설명하였다.
이 결정 방법에 의하면 단일 측정에서 눈금 간격이 큰 경우 최소 눈금의 1/10까지 어림하여 표현할 수 있고,
눈금 간격이 작은 경우 1/10까지 어림하기 불가능하다면 불확도를 1/2로 결정하여 표현할 수 있다.

따라서, 본 연구의 연필 눈금 측정 문항에서는 적어도 최소 눈금의 절반을 어림하여
표현하는 것이 유효숫자 규칙을 충실히 따른 것으로 볼 수 있다.} \\

\ganada \hly{기초 탐구과정과 통합 탐구과정

미국의 AAAS(American Association for the Advancement of Science)는
탐구과정의 중요성을 인식하고 SAPA (Science-A Process Approach)를 개발하였다.
이는 크게 기초 탐구과정(Basic Process)과 통합 탐구과정(Integrated Process)으로 분류되고
김수경 등(2004)은 기초탐구기능의 하위 영역으로 관찰, 분류, 측정, 추리, 예상하기 기능이 있고,
통합탐구기능의 하위 영역으로 문제 발견, 가설 설정, 자료 변환, 자료 해석, 변인 통제, 결론 도출, 일반화가
있음으로 번역하였다.}

\hlb{문헌 찾기} \\

\ganada \hly{측정과 자료해석

`측정'은 기초 탐구과정의 하위 영역으로
`고등학교 물리학 실험 교과서(박종찬 외. 2024)'에서 적절한 도구나 기계를 사용하여 길이, 넓이 등
단위로 나타낼 수 있는 물리량을 재어 기록하는 활동으로 표현하고 있다.
%
임청환(2011)은 측정을 관찰을 수량화하는 활동으로 표현하였고,
도구의 선택과 사용, 단위 선택, 측정 범위와 구간, 어림셈, 오차와 정확도, 반복가능성에 대한 이해가
필요함을 언급하였다.

한편, 통합 탐구과정 중 `자료 변환'은 관찰이나 측정, 실험 등으로 얻은 데이터를 기록하고,
표나 그래프 등의 여러 형태로 변환하는 활동을 말한다.
또한 측정된 여러 수치를 비교, 종합하여 의미 있는 결론을 도출하는 과정이다.
%
임청환(2011)은 자료 해석이 포괄적인 의미에서 수집한 실험 자료의 이해를 의미한다고 하였다.
본 연구에서는 반복 측정값의 평균 계산을 중심으로 학생들이 유효숫자 규칙을 적용하고
계산된 결과의 자릿수를 적절히 표현하는지를 확인하였다.} \\

\ganada 유효숫자 필요성 인식

유효숫자 필요성 인식이란 측정 과정뿐 아니라 자료 처리,
상수 적용 등 다양한 탐구 맥락에서 유효숫자가 왜 필요한지를 자각하는 정도를 의미한다.
본 연구에서는 자, 버니어캘리퍼스, 물리 상수 등 서로 다른 정밀도의 도구나 상황에 따라
학생들의 인식이 어떻게 달라지는지를 분석하였다.







