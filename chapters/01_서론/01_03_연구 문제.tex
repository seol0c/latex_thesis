\section{연구 문제}

본 연구는 과학영재 학생들이 과학 탐구 과정에서 유효숫자를 실제로 어떻게 활용하는지를
측정, 계산, 인식의 세 측면에서 분석하고자 하였다.
이에 따라 다음의 세 가지 연구 문제를 설정하였다.

\vspace*{20pt} \begin{hangblock}{4.2em}
첫째, 과학영재 학생들은 측정 도구의 최소 눈금 단위를 이용한 길이 측정 과정에서
유효숫자 규칙을 알고 있으며, 이를 실제 측정 상황에서 일관되게 적용하는가?

둘째, 과학영재 학생들은 반복 측정값의 평균을 계산할 때 유효숫자 규칙을
바르게 적용하여 자릿수를 결정하고, 계산 과정에서도 그 기준을 유지하는가?

셋째, 과학영재 학생들은 측정 도구의 정밀도나 과학적
맥락(자의 사용, 버니어캘리퍼스 활용, 물리 상수 적용)에 따라
유효숫자의 필요성을 어떻게 인식하는가?
\end{hangblock} \vspace*{20pt}
