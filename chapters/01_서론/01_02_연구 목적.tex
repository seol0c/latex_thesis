\section{연구 목적}

본 연구는 과학영재 학생들을 대상으로 측정과 자료 해석 과정에서
유효숫자 활용의 실제 양상과 그 필요성 인식 수준을 분석하여,
유효숫자 교육의 개선 방향을 제시하고자 하며, 구체적으로 다음의 목적을 가진다.

\vspace*{20pt} \begin{hangblock}{4.2em}
첫째, 측정 도구의 최소 눈금 단위를 이용한 판독 과정에서 학생들이 유효숫자 규칙을 이해하고
실제 측정 상황에 일관되게 적용하는지를 확인한다.

둘째, 반복 측정값의 평균 계산에서 유효숫자 규칙이 얼마나 정확히 적용되는지를 살펴보아,
지식 수준의 이해와 실제 계산 적용 간의 차이를 분석한다.

셋째, 다양한 탐구 맥락(자의 사용, 버니어캘리퍼스 활용, 물리 상수 적용)에서
학생들이 유효숫자의 필요성을 어떻게 인식하는지를 조사한다.
\end{hangblock} \vspace*{20pt}

최종적으로 과학영재 학생들이 유효숫자를 단순한 규칙으로 인식하는 수준을 넘어,
측정의 정밀도와 탐구의 신뢰성을 유지하는 핵심 원리로 이해하고 활용할 수 있도록 하는
교육적 방향을 제시하고자 한다.