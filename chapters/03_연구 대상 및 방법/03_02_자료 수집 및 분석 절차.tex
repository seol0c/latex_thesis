\section{자료 수집 및 분석 절차}

본 연구는 과학영재 학생들의 유효숫자 활용 양상을 측정, 계산, 인식의 세 측면에서 분석하였다.
이를 위해 다음의 세 가지 연구 주제를 중심으로 설문을 구성하였다.

\vspace*{20pt} \begin{hangblock}{4.2em}
첫째, 과학영재 학생들이 과학 탐구 과정에서 유효숫자 규칙을 실제 측정 상황에
얼마나 일관되게 적용하는지 탐색하였다.

둘째, 계산 및 자료 처리 과정에서 유효숫자 규칙을 얼마나 정확하게 활용하는지 분석하였다.

셋째, 다양한 과학 탐구 맥락에서 유효숫자의 필요성과 의미를 어떻게 인식하는지 살펴보았다.
\end{hangblock} \vspace*{20pt}

설문 문항은 이러한 세 연구 주제에 따라 측정 상황, 계산 상황, 인식 상황의 세 영역으로 구성되었으며,
각 문항은 인지적 수준(지식, 이해, 적용)과 정의적 수준(인식, 태도)을 함께 평가할 수 있도록 설계되었다. \\



%%%%%%%%%%%%%%%%%%%%%%%%%%%%%%%%%%%%%%%%%%%%%%%%%%%%%%%%%%%%%%%%%%%%%%%%%%%%%%
측정 상황(자료수집 1)은 학생들이 측정 도구의 최소 눈금 단위와 판독 규칙을
얼마나 정확히 이해하고 적용하는지를 확인하기 위한 것이다.

(1) 도구와 눈금 간격이 명시된 상황에서는 `눈금 간격이 0.1\,cm인 자를 사용할 때,
측정값은 소수점 아래 몇 자리까지 쓰는 것이 적절한가?'를 묻는 문항을 통해 이론적 판단 능력을 평가하였다.

(2) 실제 측정 장면에서는 연필의 좌측 끝이 2.00\,cm, 우측 끝이 10.67\,cm 눈금에 놓인 그림을 제시하고,
`연필의 길이는 얼마인가요?'를 묻는 방식으로 눈금의 1/10 단위를 고려한 실제 판독 능력을 점검하였다. \\



%%%%%%%%%%%%%%%%%%%%%%%%%%%%%%%%%%%%%%%%%%%%%%%%%%%%%%%%%%%%%%%%%%%%%%%%%%%%%%
계산 상황(자료수집 2)은 학생들이 평균값 계산 과정에서
유효숫자 규칙을 일관되게 적용하는지를 분석하기 위한 것이다.

(1) 도구와 눈금 간격이 명시된 상황에서는 `눈금 간격이 0.1\,cm인 자를 사용할 때
평균값을 기록한다면 소수점 아래 몇 자리까지 쓰는 것이 적절한가?'를 제시하여
규칙에 대한 인지적 이해를 확인하였다.

(2) 두 측정값(8.628\,cm, 8.625\,cm)이 주어진 문항과
(3) 아홉 개의 측정값이 제시된 문항에서는 평균값 계산 과정에서
유효숫자 규칙을 실제로 적용하는 능력을 평가하였다. \\



%%%%%%%%%%%%%%%%%%%%%%%%%%%%%%%%%%%%%%%%%%%%%%%%%%%%%%%%%%%%%%%%%%%%%%%%%%%%%%
인식 상황(자료수집 3)은 학생들이 다양한 탐구 맥락에서
유효숫자의 필요성을 어떻게 인식하는지를 파악하기 위한 것이다.

(1) 버니어캘리퍼스와 같이 정밀한 도구를 사용하는 경우,

(2) 자와 같이 정밀도가 낮은 도구를 사용하는 경우,

(3) 물리 상수를 활용한 계산의 경우를 제시하고,
각 상황에서 `유효숫자를 반드시 지켜야 한다고 생각하는가?'를 판단하게 하였다. \\

이를 이용하여 학생들이 유효숫자를 단순한 표기 규칙이 아니라
측정 도구의 정밀도와 탐구 맥락에 따라 변동하는 신뢰도의 표현으로 인식하고 있는지를 분석하였다.
수집된 응답은 문항별 정답률, 자릿수 표현 유형, 응답 경향 등을 중심으로 분석하였다.
또한 서술형 응답은 학생들이 제시한 이유와 표현을 바탕으로
유효숫자 개념의 이해 수준과 필요성 인식의 특징을 해석하였다.
%
이후 각 문항 영역별로 학생들의 유효숫자 활용 실태를 구체적으로 파악하고,
측정, 계산, 인식의 세 측면에서 비교하였다.
설문 분석 순서는 \korref{tab:설문_분석_순서}\와 같다.

%% 표. 설문 분석 순서
\SurveyProcess
