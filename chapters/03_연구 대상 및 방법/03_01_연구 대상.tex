\section{연구 대상}

본 연구의 대상은 광주광역시에 위치한 과학영재학교 2학년 재학생 74명이다.
이들은 수학, 과학 중심의 선발 과정을 거쳐 입학하였으며,
입학 이후 심화된 과학 교육과정을 이수하면서 탐구 역량을 집중적으로 강화해왔다.
%
대부분의 학생은 연구 중심의 실험 수업, R\&E 활동, 과학 탐구대회 등을 통해
과학 개념을 실제 현상과 연결하고 문제 상황을 분석하는 경험을 꾸준히 쌓아온 집단이다.
%
따라서 본 연구의 참여자들은 고등학생 수준을 넘어,
일정 수준 이상의 과학적 사고력과 탐구 수행 능력을 갖춘 과학영재로 볼 수 있다.

학생들은 「측정과 자료 해석 과정에서의 유효숫자 활용」을 주제로 한 설문에 참여하였다.
설문은 측정 도구의 최소 눈금 단위를 이용한 자릿수 결정, 평균값 계산에서의 유효숫자 규칙 적용,
다양한 측정 맥락에서의 유효숫자 필요성 인식 등 세 영역으로 구성되었다.
각 문항은 학생들의 지식 수준과 실제 적용 능력, 그리고 유효숫자 개념에 대한 인식 수준을
종합적으로 파악할 수 있도록 설계되었다.
%
설문을 이용하여 과학영재 학생들이 유효숫자를 단순한 규칙으로 인식하는 데 그치지 않고,
실제 탐구 상황에서 얼마나 일관성 있게 적용하는지를 분석하였다.