\chapter*{참고문헌}
\addcontentsline{toc}{chapter}{참고문헌}
\vspace*{4em}
\begin{hangblock}{4.2em}

\hly{박종찬 외. (2024). 고등학교 물리학 실험. 선두.}

\hly{이인호. (2020). 제15회 국제중등과학올림피아드(IJSO2018)
문항 및 우리나라 대표 학생 답안 분석
- 물리 이론 시험 문항을 중심으로 -. 국제과학영재학회지, 6(2), 77-85.}

이재봉. ``측정자료의 오차와 불확실도에 대한 학생들의 이해.''
새물리 52.5 (2006): 436-446.

이재봉 ( Jae Bong Lee ),and 이성묵 ( Sung Muk Lee ).
``학생들의 측정불확실도 개념의 결핍으로 인한 물리탐구과정에서의 어려움 분석.''
한국과학교육학회지 26.4 (2006): 581-591.

전영석. ``온도 측정 오차의 원인 및 대처 방안에 대한 과학 우수 학생의 인식 분석.''
새물리 75.1 (2025): 35-43.

지영래,and 조헌국. ``작은 질량 측정 실험 보고서에서 나타난 대학생들의 측정과 오차 이해.''
새물리 69.5 (2019): 547-558.

\hly{한국표준과학연구원. 측정표준의 중요성.
KRISS 홈페이지. \url{https://www.kriss.re.kr}} 최종 접속: 2025.11.15.



\end{hangblock}