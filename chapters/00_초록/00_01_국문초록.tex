ddtest본 연구는 과학영재 학생들이 과학 탐구 기능의 핵심 요소인 측정과 자료 해석 과정에서
유효숫자를 어떻게 적용하며, 그 적용 기준이 상황에 따라 일관되게 유지되는지를 분석하고자 하였다.
%
연구 대상은 광주광역시에 위치한 과학영재학교 2학년 재학생 74명으로,
측정과 계산, 그리고 유효숫자의 필요성 인식에 관한 설문을 실시하였다.
%
설문은 과학 탐구 기능의 단계에 따라 세 영역으로 구성하였다.
%
첫째, `측정' 영역에서는 도구의 최소 눈금 단위를 이용한 유효숫자 자리 결정 문항을 제시하여,
학생들이 측정 도구의 눈금을 1/10 단위까지 판독해야 하는 원리를 이해하고
실제로 적용할 수 있는지를 검증하였다.
%
둘째, `자료 해석' 영역에서는 반복 측정값의 평균 계산 문항을 통해
유효숫자 자릿수 규칙을 이론적으로 알고 있는 수준과 실제 계산에서
이를 일관되게 적용하는 능력의 차이를 비교하였다.
%
셋째, `인식' 영역에서는 측정 도구의 정밀도와 탐구 맥락에 따라
유효숫자의 필요성을 어떻게 판단하는지를 분석하였다.
%
연구 결과,
%
도구와 눈금 간격이 명시된 상황에서는 69\%의 학생이 유효숫자를 올바르게 처리하였으나,
실제 측정 상황에서는 8\%만이 정답을 제시하였다.
%
평균값 계산 문항에서도 이론적 선택에서는 61\%가 올바르게 응답했으나,
실제 계산에서는 20~30\%만이 정확히 적용하였다.
%
또한 유효숫자 규칙을 알고도 단순화하거나 생략하는 학생이 약 80\%에 달했으며,
자릿수를 일관되게 유지하는 학생보다 규칙을 간과하는 경향이 두드러졌다.
%
유효숫자의 필요성 인식 문항에서는 도구의 정밀도가 높을수록 필요성을 높게 인식하였으며,
버니어캘리퍼스에서는 81\%, 물리 상수는 76\%, 자 사용 상황에서는 64\%가 필요하다고 응답하였다.
그러나 일부 학생은 상수가 이미 정확한 값이라고 판단하거나,
일상생활에서는 유효숫자가 불필요하다고 인식하는 등 상황에 의존하는 판단을 보였다.
%
이러한 결과는 학생들이 유효숫자 개념을 규칙 수준에서는 이해하고 있으나,
실제 측정과 계산의 맥락에서는 그 의미와 기능을 충분히 내면화하지 못하고 있음을 시사한다.
%
따라서 유효숫자 교육은 단순한 규칙 전달을 넘어,
다양한 측정 도구와 실험 상황을 연계하여 유효숫자의 필요성과 역할을 체감하고
일관되게 적용할 수 있도록 지도하는 방향으로 보완될 필요가 있다.