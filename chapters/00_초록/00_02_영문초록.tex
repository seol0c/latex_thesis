This study aimed to analyze how gifted science students apply significant figures
during measurement and data interpretation—core components of scientific inquiry—and whether
their application criteria remain consistent across different contexts.
The participants were seventy-four second-year students attending
a science high school for gifted students in Gwangju, South Korea.
A survey was administered to assess students' understanding and use of
significant figures in measurement, calculation, and recognition of necessity.
The questionnaire consisted of three areas corresponding to stages of scientific inquiry.
First, in the measurement domain, students were asked to determine the number of decimal
places when reading scales with specified minimum divisions, in order to examine
their understanding and practical application of the rule that measurements
should be estimated to one-tenth of the smallest scale unit.
Second, in the data interpretation domain, students were presented with tasks
involving the calculation of average values from repeated measurements
to compare the consistency between their theoretical knowledge of digit rules
and their actual application during computation. Third, in the recognition domain,
students' perceptions of the necessity of significant figures were examined
in relation to the precision of measuring instruments and the scientific contexts
in which they were used.

The results showed that when scale intervals were explicitly provided,
69\% of the students responded correctly, but only 8\% produced correct
answers in actual measurement situations, indicating a substantial discrepancy
between knowledge and practice. In many cases, students reduced or omitted decimal
places despite understanding the rule, and several applied different standards
across situations. In average calculation tasks, 61\% of the students correctly
identified the number of digits to retain when given numerical data,
but only 20-31\% applied the rule correctly during computation.
Approximately 80\% of the students were classified as those who simplified
or omitted the rule even when they knew it, while only a small proportion consistently
applied significant figures throughout their responses.
Regarding recognition of necessity, students' responses varied depending
on the precision of the measuring instrument: 81\% considered significant figures
essential when using a vernier caliper, 76\% for physical constants,
and 64\% for a ruler. Some students viewed significant figures as unnecessary
in everyday measurement contexts or believed that constants were already exact values,
revealing unstable awareness of their function.

Overall, these findings indicate that gifted science students understand the rules of
significant figures conceptually but fail to internalize their meaning and apply them
consistently in real measurement and calculation contexts.
Therefore, instruction on significant figures should move beyond rote rule
transmission and instead link measurement precision and experimental context,
providing experiences that help students recognize the necessity and function
of significant figures through authentic measurement situations.
Such an approach would enable significant figures to be established not merely
as a notational rule but as a conceptual framework supporting the reliability
of scientific inquiry.