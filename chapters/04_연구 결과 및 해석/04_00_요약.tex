본 연구에서는 과학영재 학생들의 유효숫자 활용 양상을 측정, 계산, 인식의 세 측면에서 분석하였다.

\vspace*{20pt} \begin{hangblock}{4.2em}
첫째, 측정 상황에서는 측정 도구의 최소 눈금 단위를 이용하여 학생들이
유효숫자 규칙을 실제 측정 상황에 얼마나 일관되게 적용하는지를 확인하였다.

둘째, 계산 상황에서는 평균값 계산 문항을 통해 학생들이 유효숫자 규칙을
계산 및 자료 처리 과정에서 얼마나 정확히 활용하는지를 분석하였다.

셋째, 인식 상황에서는 측정 도구의 정밀도와 과학 탐구 맥락을 달리하여
제시된 문항을 통해 학생들이 유효숫자의 필요성과 의미를 어떻게 인식하는지를 살펴보았다.

\end{hangblock} \vspace*{20pt}

이 세 영역의 결과를 종합적으로 분석함으로써, 과학영재 학생들의 유효숫자 활용이
지식적 이해에 머무는지, 아니면 실제 탐구 과정에서 일관된 실천으로 이어지는지를 탐색하였다.