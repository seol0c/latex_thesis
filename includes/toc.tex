% 목차 스타일 조정
\makeatletter

% toc 안의 폰트 크기 등 기존 명령 무시
\def\TOC@killfonts{%
  \let\fontsize\@gobbletwo
  \let\selectfont\relax}

% Chapter
\renewcommand{\l@chapter}[2]{%
  \vskip 0.8em % 새 로마자 시작 전에 한 줄 띄우기
  {\TOC@killfonts
    \def\numberline##1{{\normalfont\normalsize ##1.\,\,}}%
    \@dottedtocline{0}{0em}{1.4em}{#1}{#2}}}

% Chapterstar는 위 여백 주지 않음
\newcommand{\l@chapterstar}[2]{%
  {\TOC@killfonts
    \def\numberline##1{}%
    \@dottedtocline{0}{0em}{0em}{#1}{#2}}}
\providecommand*{\toclevel@chapterstar}{0}

% Section
\renewcommand{\l@section}[2]{{
    \TOC@killfonts
    \def\numberline##1{{\normalfont\normalsize ##1\,\,}}%
    \@dottedtocline{1}{1.2em}{1.4em}{#1}{#2}}}

% Subsection
\renewcommand{\l@subsection}[2]{{%
    \TOC@killfonts
    \def\numberline##1{{\normalfont\normalsize ##1\,\,}}%
    \@dottedtocline{2}{3.8em}{3.2em}{#1}{#2}}}

% 점선 간격 조절
\renewcommand{\@dotsep}{1}

% 목차 깊이
\setcounter{tocdepth}{1}

% 목차 제목
\renewcommand{\contentsname}{}
\chapter*{\hypertarget{toc}{}목 ~ ~ ~ 차}
\vspace{-2em}

% 목차 출력 (chapter 번호 충돌 방지)
\begingroup
  \let\chapter\section
  \tableofcontents
\endgroup
%
\makeatother