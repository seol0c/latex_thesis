

%%%%%%%%%%%%%%%%%%%%%%%%%%%%%%%%%%%%%%%%%%%%%%%%%%%%%%%%%%%%%%%%%%%%%%%%%%%%%%%%%%%%%%%%%%%%%
%% 표. 설문 분석 순서
%%%%%%%%%%%%%%%%%%%%%%%%%%%%%%%%%%%%%%%%%%%%%%%%%%%%%%%%%%%%%%%%%%%%%%%%%%%%%%%%%%%%%%%%%%%%%
\newcommand{\SurveyProcess}{
\vspace*{8pt}
\begin{table}[h!]
\centering
\caption{설문 분석 순서}
\label{tab:설문_분석_순서}
\vspace*{-8pt}
{\fontsize{9pt}{9pt}\selectfont\begin{tabular}{
>{\centering\arraybackslash}m{0.4cm} |
  >{\centering\arraybackslash}m{1.8cm} |
  >{\centering\arraybackslash}m{4.3cm} |
  >{\centering\arraybackslash}m{5.5cm}}\hline
\rowcolor{gray!20}
\shortstack[1]{\rule{0pt}{1.4em}순 \\[4pt]서} & \shortstack[1]{\rule{0pt}{1.4em}인지적, \\[4pt] 정의적 수준} & 탐색 주제 & 질문 내용 \\ \hline
\multirow{2}{*}{\raisebox{-2\height}{1}}
 & 지식, 이해
 & \shortstack[1]{\rule{0pt}{1.4em}측정에서 도구의 눈금을 \\[4pt] 1/10까지 판독해야 하며, \\[4pt] 적용해야 함을 아는가?}
 & \shortstack[1]{\rule{0pt}{1.4em}눈금 간격이 0.1\,cm인 자를 사용할 때, \\[4pt] 측정값은 소수점 아래 몇 자리까지 \\[4pt] 쓰는 것이 적절한지 판단하기}\\
\cline{2-4}
 & 적용
 & 실제로 측정값을 읽을 때 측정 도구의 1/10까지 고려하는가?
 & 연필의 좌측과 우측 끝 눈금을 읽고 길이를 계산하기\\ \hline
\multirow{2}{*}{\raisebox{-2\height}{2}}
 & 지식, 이해
 & 자료 해석에서 유효숫자 규칙을 바르게 알고, 적용해야 함을 아는가?
 & 측정값이 주어질 때, 평균값은 소수점 아래 몇 자리까지 기록하는 것이 적절한지 판단하기\\
\cline{2-4}
 & 적용
 & 주어진 측정값에서 유효숫자 규칙에 따라 평균을 계산하는가?
 & \shortstack[1]{\rule{0pt}{1.4em}(1) 두 측정값의 평균 계산 \\[4pt] (2) 여러 측정값의 평균 계산}\\ \hline
3 & 인식(태도)
 & 다양한 탐구 맥락에서 유효숫자 적용의 필요성을 인식하는가?
 & \shortstack[1]{\rule{0pt}{1.4em}(1)자 이용 측정 필요성 판단 \\[4pt] (2) 버니어캘리퍼스 필요성 판단 \\[4pt] (3) 물리 상수 계산 필요성 판단}\\ \hline
\end{tabular}}\vspace*{0pt}\end{table}}


%%%%%%%%%%%%%%%%%%%%%%%%%%%%%%%%%%%%%%%%%%%%%%%%%%%%%%%%%%%%%%%%%%%%%%%%%%%%%%%%%%%%%%%%%%%%%
%% 표. 도구와 눈금 간격이 명시된 상황에서 측정값의 자릿수 선택 응답
%%%%%%%%%%%%%%%%%%%%%%%%%%%%%%%%%%%%%%%%%%%%%%%%%%%%%%%%%%%%%%%%%%%%%%%%%%%%%%%%%%%%%%%%%%%%%
\newcommand{\DigitChoiceWithScaleInfo}{%
\vspace*{8pt}
\begin{table}[h!]
\centering
\caption{도구와 눈금 간격이 명시된 상황에서 측정값의 자릿수 선택 응답}
\label{tab:측정_자릿수_선택_응답}
\vspace*{-8pt}
{\fontsize{9pt}{9pt}\selectfont
\begin{tabular}{
>{\centering\arraybackslash}m{1.6cm} |
>{\centering\arraybackslash}m{2.2cm}} \hline
\rowcolor{gray!20}
응답 & 비율 (학생 수) \\ \hline
* 2 & 69\% (51명) \\ \hline
1 & 23\% (17명) \\ \hline
3 또는 4 & 7\% (5명) \\ \hline
\end{tabular}} \\
\vspace*{4pt}
{\footnotesize *답은 2이다.}
\vspace*{0pt}\end{table}}


%%%%%%%%%%%%%%%%%%%%%%%%%%%%%%%%%%%%%%%%%%%%%%%%%%%%%%%%%%%%%%%%%%%%%%%%%%%%%%%%%%%%%%%%%%%%%
%% 표. 실제 측정 상황에서 전체 학생 응답 구분
%%%%%%%%%%%%%%%%%%%%%%%%%%%%%%%%%%%%%%%%%%%%%%%%%%%%%%%%%%%%%%%%%%%%%%%%%%%%%%%%%%%%%%%%%%%%%
\newcommand{\ActualMeasurementType}{%
\vspace*{8pt}
\begin{table}[h!]
\centering
\caption{실제 측정 상황에서 전체 학생 응답 구분}
\label{tab:실제_측정_상황에서_전체_학생_응답_구분}
\vspace*{-8pt}
{\fontsize{9pt}{9pt}\selectfont
\begin{tabular}{
>{\centering\arraybackslash}m{1.4cm} |
>{\centering\arraybackslash}m{1.2cm} |
>{\centering\arraybackslash}m{4.0cm} |
>{\centering\arraybackslash}m{1.8cm} |
>{\centering\arraybackslash}m{2.0cm}}
\hline
\rowcolor{gray!20}
\multicolumn{3}{c|}{\diagbox[width=7.8cm,height=1.2cm]{구분}{응답}}
& 사용한 자릿수 
& 비율 (학생수) \\ \hline
\multirow{2}{*}{일관형}
& *A1 & 소수 둘째 자리 사용 & 2-2-2 & 8\% (6명) \\ \cline{2-5}
& A2  & 소수 첫째 자리 사용 & 1-1-1 & 4\% (3명) \\ \hline
\multirow{2}{*}{간과형}
& B1 & 첫 항목만 정수처리 & 0-2-2 & 36\% (27명) \\ \cline{2-5}
& B2 & 첫 항목만 정수처리 & 0-1-1 & 46\% (34명) \\ \hline
불일치형 & C & 일관성 없음 & 2-1-2 등 & 5\% (4명) \\ \hline
\end{tabular}} \\
\vspace*{4pt}
{\footnotesize *답은 A1유형(2-2-2)이다.}
\vspace*{0pt}\end{table}}


%%%%%%%%%%%%%%%%%%%%%%%%%%%%%%%%%%%%%%%%%%%%%%%%%%%%%%%%%%%%%%%%%%%%%%%%%%%%%%%%%%%%%%%%%%%%%
%% 표. 도구와 눈금 간격이 명시된 상황에서 평균값 자릿수 선택 응답
%%%%%%%%%%%%%%%%%%%%%%%%%%%%%%%%%%%%%%%%%%%%%%%%%%%%%%%%%%%%%%%%%%%%%%%%%%%%%%%%%%%%%%%%%%%%%
\newcommand{\DigitChoiceForMean}{%
\vspace*{8pt}
\begin{table}[h!]
\centering
\caption{도구와 눈금 간격이 명시된 상황에서 평균값 자릿수 선택 응답}
\label{tab:평균_자릿수_선택_응답}
\vspace*{-8pt}
{\fontsize{9pt}{9pt}\selectfont
\begin{tabular}{
>{\centering\arraybackslash}m{1.6cm} |
>{\centering\arraybackslash}m{2.2cm}} \hline
\rowcolor{gray!20}
응답 & 비율 (학생 수) \\ \hline
* 3 & 61\% (45명) \\ \hline
2 또는 4 & 39\% (29명) \\ \hline
\end{tabular}} \\
\vspace*{4pt}
{\footnotesize *답은 2이다.}
\vspace*{0pt}\end{table}}


%%%%%%%%%%%%%%%%%%%%%%%%%%%%%%%%%%%%%%%%%%%%%%%%%%%%%%%%%%%%%%%%%%%%%%%%%%%%%%%%%%%%%%%%%%%%%
%% 표. 평균값 구하기 문제에서 학생들의 정답률
%%%%%%%%%%%%%%%%%%%%%%%%%%%%%%%%%%%%%%%%%%%%%%%%%%%%%%%%%%%%%%%%%%%%%%%%%%%%%%%%%%%%%%%%%%%%%
\newcommand{\MeanCalculationAccuracy}{%
\vspace*{8pt}
\begin{table}[h!]
\centering
\caption{평균값 구하기 문제에서 학생들의 정답률}
\label{tab:평균값_구하기_문제에서_학생들의_정답률}
\vspace*{-8pt}
{\fontsize{9pt}{9pt}\selectfont
\begin{tabular}{
>{\centering\arraybackslash}m{3cm} |
>{\centering\arraybackslash}m{3cm}} \hline
\rowcolor{gray!20}
문항 & 비율 (정답 학생 수) \\ \hline
두 측정값의 평균 & 20\% (15명) \\ \hline
여러 값의 평균 & 31\% (23명) \\ \hline
\end{tabular}}
\vspace*{0pt}\end{table}}


%%%%%%%%%%%%%%%%%%%%%%%%%%%%%%%%%%%%%%%%%%%%%%%%%%%%%%%%%%%%%%%%%%%%%%%%%%%%%%%%%%%%%%%%%%%%%
%% 표. 여러 탐구 상황에서 유효숫자 필요성 응답률
%%%%%%%%%%%%%%%%%%%%%%%%%%%%%%%%%%%%%%%%%%%%%%%%%%%%%%%%%%%%%%%%%%%%%%%%%%%%%%%%%%%%%%%%%%%%%
\newcommand{\SigFigNecessityResponses}{%
\vspace*{8pt}
\begin{table}[h!]
\centering
\caption{여러 탐구 상황에서 유효숫자 필요성 응답률}
\label{tab:여러_탐구_상황에서_유효숫자_필요성_응답률}
\vspace*{-8pt}
{\fontsize{9pt}{9pt}\selectfont
\begin{tabular}{
>{\centering\arraybackslash}m{3cm} |
>{\centering\arraybackslash}m{3cm}} \hline
\rowcolor{gray!20}
항목 & 비율 (응답 학생 수) \\ \hline
자 & 64\% (47명) \\ \hline
버니어캘리퍼스 & 81\% (60명) \\ \hline
물리상수 & 76\% (56명) \\ \hline
\end{tabular}}
\vspace*{0pt}\end{table}}


%%%%%%%%%%%%%%%%%%%%%%%%%%%%%%%%%%%%%%%%%%%%%%%%%%%%%%%%%%%%%%%%%%%%%%%%%%%%%%%%%%%%%%%%%%%%%
%% 표. 측정과 자료 해석에서 정답률 차이
%%%%%%%%%%%%%%%%%%%%%%%%%%%%%%%%%%%%%%%%%%%%%%%%%%%%%%%%%%%%%%%%%%%%%%%%%%%%%%%%%%%%%%%%%%%%%
\newcommand{\MeasurementVsInterpretationGap}{%
\vspace*{8pt}
\begin{table}[h!]
\centering
\caption{측정과 자료 해석에서 정답률 차이}
\label{tab:측정과_자료_해석에서_정답률_차이}
\vspace*{-8pt}
{\fontsize{9pt}{9pt}\selectfont
\begin{tabular}{
>{\centering\arraybackslash}m{7cm} |
>{\centering\arraybackslash}m{1.8cm} |
>{\centering\arraybackslash}m{1.8cm} |
>{\centering\arraybackslash}m{1.8cm}} \hline
\rowcolor{gray!20}
& \multicolumn{3}{c}{\cellcolor{gray!20}비율 (학생수)} \\ \cline{2-4}
\rowcolor{gray!20}
\smash{\raisebox{7pt}{항목}} & 설명된 상황 & \multicolumn{2}{c}{실제 적용 상황} \\ \hline
최소 눈금 단위를 이용한 유효숫자 자리 결정 & 69\% (51명) & \multicolumn{2}{c}{8\% (6명)} \\ \hline
여러 값의 평균 구하기 & 61\% (45명) & 20\% (15명) & 31\% (23명) \\ \hline
\end{tabular}}
\vspace*{0pt}\end{table}}

