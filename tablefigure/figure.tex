
%%%%%%%%%%%%%%%%%%%%%%%%%%%%%%%%%%%%%%%%%%%%%%%%%%%%%%%%%%%%%%%%%%%%%%%%%%%%%%%%%%%%%%%%%%%%%
%% 그림. 자의 눈금 판독 문항
%%%%%%%%%%%%%%%%%%%%%%%%%%%%%%%%%%%%%%%%%%%%%%%%%%%%%%%%%%%%%%%%%%%%%%%%%%%%%%%%%%%%%%%%%%%%%
\newcommand{\rulerpencil}{ % 명령 이름
\vspace*{10pt}
\begin{figure}[h!] \centering
\caption{자의 눈금 판독 문항} % 캡션(파일명과 동일)
\vspace*{-8pt}
\includegraphics[scale=0.24] % 스케일
{../tablefigure/figures/rulerpencil} \\ % 파일 경로(파일명)
\vspace*{-10pt} % 아래 추가 메모 필요시
{\footnotesize (연필의 좌측 끝이 2.00\,cm 눈금에 걸쳐 있음)}
\label{fig:rulerpencil}\vspace*{0pt}\end{figure}}

%%%%%%%%%%%%%%%%%%%%%%%%%%%%%%%%%%%%%%%%%%%%%%%%%%%%%%%%%%%%%%%%%%%%%%%%%%%%%%%%%%%%%%%%%%%%%
%% 그림. 버니어캘리퍼스를 이용한 연필 길이 측정
%%%%%%%%%%%%%%%%%%%%%%%%%%%%%%%%%%%%%%%%%%%%%%%%%%%%%%%%%%%%%%%%%%%%%%%%%%%%%%%%%%%%%%%%%%%%%
\newcommand{\caliperpencil}{
\vspace*{10pt}
\begin{figure}[h!] \centering
\caption{버니어캘리퍼스를 이용한 연필 길이 측정}
\vspace*{-8pt}
\includegraphics[scale=0.24]
{../tablefigure/figures/caliperpencil}
\label{fig:caliperpencil}\vspace*{0pt}\end{figure}}
