%%%%%%%%%%%%%%%%%%%%%%%%%%%%%%%%%%%%%%%% 패키지, 사용자 정의 매크로 호출 
% 문서 초기 설정
\usepackage{geometry} % 페이지 용지 설정

% 공통
\usepackage[hidelinks]{hyperref} % 참조나 목차에서 하이퍼링크가 가능하게 함
\usepackage{xifthen} % if 조건문 사용
\usepackage{silence} % 경고 제거

% 폰트, 서식
\usepackage{luatexko} % 한글 폰트 사용. kotex의 lua 전용 후속 버전
\usepackage[tracking=true]{microtype} % 자간

% 색상
\usepackage{xcolor} % 색상

% 이미지
\usepackage{graphicx} % 이미지 삽입

% 페이지
\usepackage{fancyhdr} % 머리말, 꼬리말
\usepackage{import} % main이 chapter 호출, cahpter가 section 호출(main에 대해 상대경로 사용)
\usepackage{titlesec} % 타이틀(chapter) 형태 변경

% 캡션, 링크
\usepackage{caption} % 캡션 스타일 조정

%% 수식

% 표, 디자인
\usepackage[most]{tcolorbox} % 박스 만들기 - box.sty에서 필수
\usepackage{float} % 표/그림 위치 고정
\usepackage{multirow} % 표에서 여러 행 병합
\usepackage{tikz} % 사각박스, 선긋기(교사용 메모 박스)
\usetikzlibrary{calc} % tikz에서 cal 사용
\usepackage{tikzpagenodes} % 사각박스, 선긋기(목차로 이동하는 박스)
\usepackage{colortbl}     % 셀 색칠
\usepackage{diagbox} % 표에서 대각선 그을 때 필요함
%%%%%%%%%%%%%%%%%%%%%%%%%%%%%%%%%%%%%%%% 텍스트 스타일 - 볼드, 이탤릭
% 한글 폰트에 볼드체가 없는 경우가 있음. 기본적으로 bold체로 바꾸고, 한글은 FakeBold, 강도를 숫자로 조정
\newcommand{\bd}[1]{\ifmmode{\let\txt\text \def\text##1{\txt{\addhangulfontfeatures{AutoFakeBold=3}\textbf{##1}}}\bm{#1}}%
\else{\addhangulfontfeatures{AutoFakeBold=3}\textbf{\boldmath #1}}\fi}

% 한글 폰트에 이탤릭체가 대부분 없음. 기본적으로 이탤릭체로 바꾸고, 한글은 FakeSlant, 강도를 숫자로 조정
\newcommand{\itl}[1]{{\addhangulfontfeatures{AutoFakeSlant=.3}\textit{#1}}}

%%%%%%%%%%%%%%%%%%%%%%%%%%%%%%%%%%%%%%%% 색상, 형광펜
%% 색상 정의 % xcolor, transparent 필요
\definecolor{myred}{RGB}{255,70,70}
\definecolor{myyellow}{RGB}{255,230,40}
\definecolor{mylime}{RGB}{160,240,120}
\definecolor{myblue}{RGB}{70,90,230}
\definecolor{mywhite}{RGB}{255,255,255}

%% 색상 텍스트
\newcommand{\red}[1]{\textcolor{myred}{#1}}
\newcommand{\yellow}[1]{\textcolor{myyellow}{#1}}
\newcommand{\lime}[1]{\textcolor{mylime}{#1}}
\newcommand{\blue}[1]{\textcolor{myblue}{#1}}
\newcommand{\white}[1]{\textcolor{mywhite}{#1}}

%% 형광펜, 강조 스타일 정의
\newcommand{\hlr}[1]{\highLight[myred]{#1}}
\newcommand{\hly}[1]{\highLight[myyellow]{#1}}
\newcommand{\hll}[1]{\highLight[mylime]{#1}}
\newcommand{\hlb}[1]{\highLight[myblue]{#1}}
\newcommand{\hlw}[1]{\highLight[mywhite]{#1}}

%% seolmode가 4일 때는 \hll(라임), 나머지는 \hlw - 똑같이 여백 차지
\newcount\seolmode
\newcommand{\seoli}[1]{%
  \ifnum\seolmode=4 \hll{#1}%
  \else \hlw{#1}\fi}
% Batang은 Bold 없으므로 BoldFont를 Batang으로 매핑
\setmainhangulfont{Batang}[BoldFont=Batang]

% 강제 볼드
\newcommand{\TitleFakeBold}{\addhangulfontfeatures{AutoFakeBold=3}}


%% 여백 설정
% Chapter
\renewcommand{\thechapter}{\fontsize{18pt}{20pt}\selectfont\Roman{chapter}}
\titleformat{\chapter}
  {\normalfont\fontsize{16pt}{18pt}\bfseries\centering\TitleFakeBold} % 폰트 사이즈, 가운데정렬
  {\thechapter.} % Ⅰ.
  {0.4em} % Ⅰ. 다음 여백
  {}
\titlespacing{\chapter}{0pt}{-16pt}{0pt} % 좌, 위, 아래 여백

% chapter*
\titleformat{name=\chapter,numberless}
  {\normalfont\fontsize{16pt}{18pt}\bfseries\centering\TitleFakeBold}
  {}
  {0pt}
  {}

% Section
\renewcommand{\thesection}{\fontsize{14pt}{16pt}\selectfont\arabic{section}.}
\titleformat{\section}
  {\normalfont\fontsize{13pt}{15pt}\bfseries\TitleFakeBold}
  {\thesection}
  {0.4em}
  {} 
\titlespacing{\section}{0pt}{0pt}{0pt}

% 페이지번호 중앙 및 대시 표현
\makeatletter
\def\ps@plain{%
  \let\@oddhead\@empty
  \let\@evenhead\@empty
  \renewcommand{\@oddfoot}{\hfil - \thepage\ - \hfil}%
  \renewcommand{\@evenfoot}{\hfil - \thepage\ - \hfil}}%
\def\ps@headings{\ps@plain}
\def\ps@myheadings{\ps@plain}
\makeatother
\setlength{\footskip}{\dimexpr\footskip+1.4em\relax}% 아래로 내림
\pagestyle{plain} % 적용


%%%%%%%%%%%%%%%%%%%%%%%%%%%%%%%%%%%%%%%% 페이지 설정
\newcommand{\geoTop}   {35mm} % 위쪽 여백
\newcommand{\geoBottom}{40mm} % 아래 여백
\newcommand{\geoLeft}  {33mm} % 좌측 여백
\newcommand{\geoRight} {33mm} % 우측 여백
\geometry{a4paper,top=\geoTop, bottom=\geoBottom, left=\geoLeft, right=\geoRight}

\setmainhangulfont{Batang}[LetterSpace=10] % 한글 자간 늘림
\setmainfont{Times New Roman}  % 영어는 기본 간격
\linespread{1.7} % 줄간격
\setlength{\parindent}{1.4em} % 들여쓰기

\raggedbottom % 글 자동배치 하지 않고 단락, 줄간격 조절하지 않음

% 페이지 하단의 꼬릿말 위치에 하이퍼링크(목차로 이동) 박스 추가
\newcommand{\linkbox}{\AddToHook{shipout/background}{%
\begin{tikzpicture}[remember picture, overlay]
\node[anchor=south west, inner sep=0pt] at ([yshift=0mm]current page.south west) { % y축방향 이동
\hyperlink{toc}{\phantom{\rule{\paperwidth}{3mm}}}}; % 이동박스 두께
\end{tikzpicture}}}

% 정해진 폭 안에서 글자 사이를 균등 분배(양쪽정렬)
\ExplSyntaxOn
\NewDocumentCommand{\JustifyName}{ O{6cm} m }{\makebox[#1][s]{%
\seq_set_split:Nnn \l_tmpa_seq {} {#2}% 글자 분해
\seq_use:Nn \l_tmpa_seq { \hfill }% 글자 사이에 \hfill 삽입
}}\ExplSyntaxOff

% 첫째, 둘째, 참고문헌 들여쓰기용 block
\makeatletter
\newenvironment{hangblock}[2][1]{%
  \par
  \hangafter=#1\relax
  \hangindent=#2\relax
  \everypar{\hangafter=#1\relax\hangindent=#2\relax}}{\par\everypar{}}
\makeatother

% 표 라벨 형식
\DeclareCaptionLabelFormat{tablelabel}{<\,#1\ #2\,>\,\,}
\renewcommand{\thetable}{\arabic{table}}
\captionsetup[table]{name=표, labelformat=tablelabel, labelsep=space, textformat=simple}

% 그림 라벨 형식
\DeclareCaptionLabelFormat{figurelabel}{<\,#1\ #2\,>\,\,}
\renewcommand{\thefigure}{\arabic{figure}}
\captionsetup[figure]{name=그림, labelformat=figurelabel, labelsep=space, textformat=simple}

% 폰트 크기와 위아래 여백
\DeclareCaptionFont{mycaptionfont}{\fontsize{9pt}{9pt}\selectfont}
\captionsetup[figure]{font=mycaptionfont}
\captionsetup[table]{font=mycaptionfont}

% 챕터별이 아닌 본문 전체로 표 번호 부여
\counterwithout{table}{chapter}
\counterwithout{figure}{chapter}

% autoref 참조 이름 재정의
\renewcommand*{\tableautorefname}{표}
\renewcommand*{\figureautorefname}{그림}

% autoref 출력 포맷 재정의 - korref로 지정
\makeatletter\newcommand{\korref}[1]{<\,\autoref{#1}\,>}\makeatother

%%%%%%%%%%%%%%%%%%%%%%%%%%%%%%%%%%%%%%%% 경고 무시
\hbadness=10000 % underfull hbox 경고 무시(수평 방향 여백 경고 badness 10000까지 무시)
\vbadness=10000 % overfull hbox 경고 무시(수직 방향 페이니 나눔 단락 경고 badness 10000까지 무시)
\WarningFilter{latex}{Command \showhyphens}

% bold 없는 폰트에서 bold 요청할 때 경고 제거
\makeatletter
\DeclareFontShape{TU}{Batang(0)}{b}{n}{<-> ssub * Batang(0)/m/n}{}
\DeclareFontShape{TU}{Batang(1)}{b}{n}{<-> ssub * Batang(1)/m/n}{}
\makeatother

% 가나다라 카운터 ganada
\newcounter{ganadacnt}
\newcommand{\ganadareset}{\setcounter{ganadacnt}{0}}
\newcommand{\ganada}{%
  \ifnum\value{ganadacnt}>13
    \setcounter{ganadacnt}{0}\fi
  \ifcase\value{ganadacnt}
    가.\ \or 나.\ \or 다.\ \or 라.\ \or 마.\ \or 바.\ \or 사.\ \or 아.\ \or 자.\ \or 차.\ \or 카.\ \or 타.\ \or 파.\ \or 하.\ %
  \fi\stepcounter{ganadacnt}}